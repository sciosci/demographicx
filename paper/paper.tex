\PassOptionsToPackage{unicode=true}{hyperref} % options for packages loaded elsewhere
\PassOptionsToPackage{hyphens}{url}
%
\documentclass[]{article}
\usepackage{lmodern}
\usepackage{amssymb,amsmath}
\usepackage{ifxetex,ifluatex}
\usepackage{fixltx2e} % provides \textsubscript
\ifnum 0\ifxetex 1\fi\ifluatex 1\fi=0 % if pdftex
  \usepackage[T1]{fontenc}
  \usepackage[utf8]{inputenc}
  \usepackage{textcomp} % provides euro and other symbols
\else % if luatex or xelatex
  \usepackage{unicode-math}
  \defaultfontfeatures{Ligatures=TeX,Scale=MatchLowercase}
\fi
% use upquote if available, for straight quotes in verbatim environments
\IfFileExists{upquote.sty}{\usepackage{upquote}}{}
% use microtype if available
\IfFileExists{microtype.sty}{%
\usepackage[]{microtype}
\UseMicrotypeSet[protrusion]{basicmath} % disable protrusion for tt fonts
}{}
\IfFileExists{parskip.sty}{%
\usepackage{parskip}
}{% else
\setlength{\parindent}{0pt}
\setlength{\parskip}{6pt plus 2pt minus 1pt}
}
\usepackage{hyperref}
\hypersetup{
            pdftitle={demographicx: A Python package for estimating gender and ethnicity using deep learning transformers},
            pdfborder={0 0 0},
            breaklinks=true}
\urlstyle{same}  % don't use monospace font for urls
\usepackage{color}
\usepackage{fancyvrb}
\newcommand{\VerbBar}{|}
\newcommand{\VERB}{\Verb[commandchars=\\\{\}]}
\DefineVerbatimEnvironment{Highlighting}{Verbatim}{commandchars=\\\{\}}
% Add ',fontsize=\small' for more characters per line
\newenvironment{Shaded}{}{}
\newcommand{\AlertTok}[1]{\textcolor[rgb]{1.00,0.00,0.00}{\textbf{#1}}}
\newcommand{\AnnotationTok}[1]{\textcolor[rgb]{0.38,0.63,0.69}{\textbf{\textit{#1}}}}
\newcommand{\AttributeTok}[1]{\textcolor[rgb]{0.49,0.56,0.16}{#1}}
\newcommand{\BaseNTok}[1]{\textcolor[rgb]{0.25,0.63,0.44}{#1}}
\newcommand{\BuiltInTok}[1]{#1}
\newcommand{\CharTok}[1]{\textcolor[rgb]{0.25,0.44,0.63}{#1}}
\newcommand{\CommentTok}[1]{\textcolor[rgb]{0.38,0.63,0.69}{\textit{#1}}}
\newcommand{\CommentVarTok}[1]{\textcolor[rgb]{0.38,0.63,0.69}{\textbf{\textit{#1}}}}
\newcommand{\ConstantTok}[1]{\textcolor[rgb]{0.53,0.00,0.00}{#1}}
\newcommand{\ControlFlowTok}[1]{\textcolor[rgb]{0.00,0.44,0.13}{\textbf{#1}}}
\newcommand{\DataTypeTok}[1]{\textcolor[rgb]{0.56,0.13,0.00}{#1}}
\newcommand{\DecValTok}[1]{\textcolor[rgb]{0.25,0.63,0.44}{#1}}
\newcommand{\DocumentationTok}[1]{\textcolor[rgb]{0.73,0.13,0.13}{\textit{#1}}}
\newcommand{\ErrorTok}[1]{\textcolor[rgb]{1.00,0.00,0.00}{\textbf{#1}}}
\newcommand{\ExtensionTok}[1]{#1}
\newcommand{\FloatTok}[1]{\textcolor[rgb]{0.25,0.63,0.44}{#1}}
\newcommand{\FunctionTok}[1]{\textcolor[rgb]{0.02,0.16,0.49}{#1}}
\newcommand{\ImportTok}[1]{#1}
\newcommand{\InformationTok}[1]{\textcolor[rgb]{0.38,0.63,0.69}{\textbf{\textit{#1}}}}
\newcommand{\KeywordTok}[1]{\textcolor[rgb]{0.00,0.44,0.13}{\textbf{#1}}}
\newcommand{\NormalTok}[1]{#1}
\newcommand{\OperatorTok}[1]{\textcolor[rgb]{0.40,0.40,0.40}{#1}}
\newcommand{\OtherTok}[1]{\textcolor[rgb]{0.00,0.44,0.13}{#1}}
\newcommand{\PreprocessorTok}[1]{\textcolor[rgb]{0.74,0.48,0.00}{#1}}
\newcommand{\RegionMarkerTok}[1]{#1}
\newcommand{\SpecialCharTok}[1]{\textcolor[rgb]{0.25,0.44,0.63}{#1}}
\newcommand{\SpecialStringTok}[1]{\textcolor[rgb]{0.73,0.40,0.53}{#1}}
\newcommand{\StringTok}[1]{\textcolor[rgb]{0.25,0.44,0.63}{#1}}
\newcommand{\VariableTok}[1]{\textcolor[rgb]{0.10,0.09,0.49}{#1}}
\newcommand{\VerbatimStringTok}[1]{\textcolor[rgb]{0.25,0.44,0.63}{#1}}
\newcommand{\WarningTok}[1]{\textcolor[rgb]{0.38,0.63,0.69}{\textbf{\textit{#1}}}}
\usepackage{longtable,booktabs}
% Fix footnotes in tables (requires footnote package)
\IfFileExists{footnote.sty}{\usepackage{footnote}\makesavenoteenv{longtable}}{}
\setlength{\emergencystretch}{3em}  % prevent overfull lines
\providecommand{\tightlist}{%
  \setlength{\itemsep}{0pt}\setlength{\parskip}{0pt}}
\setcounter{secnumdepth}{0}
% Redefines (sub)paragraphs to behave more like sections
\ifx\paragraph\undefined\else
\let\oldparagraph\paragraph
\renewcommand{\paragraph}[1]{\oldparagraph{#1}\mbox{}}
\fi
\ifx\subparagraph\undefined\else
\let\oldsubparagraph\subparagraph
\renewcommand{\subparagraph}[1]{\oldsubparagraph{#1}\mbox{}}
\fi

% set default figure placement to htbp
\makeatletter
\def\fps@figure{htbp}
\makeatother


\title{\texttt{demographicx}: A Python package for estimating gender and
ethnicity using deep learning transformers}
    \usepackage{authblk}
                                        \author[]{Lizhen Liang and Daniel E. Acuna}
                                                            \affil{School of Information Studies, Syracuse University, Syracuse, NY}
                                            \date{}

\begin{document}
\maketitle

\hypertarget{abstract}{%
\section{Abstract}\label{abstract}}

Plenty of research questions would benefit from understanding whether
demographic factors are associated with social phenomena. Accessing this
information from individuals is many times infeasible or unethical.
While software packages have been developed for inferring this
information, they are often untested, outdated, or with licensing
restrictions. Here, we present a Python package to infer the gender and
ethnicity of individuals using first names or full names. We employ a
deep learning transformer of text fragments based on BERT to fine-tune a
network. We train our model on Torkiv (Torvik 2018), and extensively
validate our predictions. Our gender prediction achieves an average F1
of 0.942 across female, male, and unknown gender names. Similarly, our
ethnicity prediction achieves an average F1 of 0.94 across White, Black,
Hispanic, and Asian categories. We hope that by making our package open
and tested, we improve demographic estimates for many research fields
that are trying to understand these factors.

\hypertarget{statement-of-need}{%
\subsection{Statement of Need}\label{statement-of-need}}

Demographic information such as gender and ethnicity is a crucial
dimension to understand many social phenomena. Gender and ethnicity are
of course only a fraction of the critical factors that should be
analyzed about individuals (see (Acuna 2020)), yet they have attracted
increased interest from the research community. In social science, for
example, it has been shown that gender and race are important for
scientific collaboration (Larivière et al. 2013), mentorship (Schwartz,
Liénard, and David 2021), and funding (Ginther et al. 2011). Accessing
this information is, however, challenging because of legal or ethical
reasons. Many studies resort to analyzing names to make these kinds of
inferences, but the packages and services they often use are
non-reproducible or rely on proprietary information with unknown methods
and validations (e.g., genderize.io). Without access to an easy-to-use,
public, open, and validated method, we risk making inferences about
these kinds of phenomena without good grounding. While inferring
demographics from names has potential flaws (Kozlowski et al. 2021), it
is sometimes the only input we have; it is desirable to have better
algorithms than the ones currently available.

\begin{longtable}[]{@{}lllllll@{}}
\caption{Gender prediction performance on validation split of the mixed
data set and Social Security Administration (SSA) popular newborn names.
Names in SSA that are also in validation and with a ``unknown'' label in
authori-ty data set will take the label from authori-ty in order to
validate performance on ``unknown'' class.}\tabularnewline
\toprule
\endhead
& \textbf{Male} & & \textbf{Female} & & \textbf{Unknown}
&\tabularnewline
& Validation & SSA & Validation & SSA & Validation & SSA\tabularnewline
\textbf{F1} & 0.961 & 0.813 & 0.975 & 0.915 & 0.889 &
0.504\tabularnewline
\textbf{Acc} & 0.972 & 0.711 & 0.979 & 0.885 & 0.862 &
0.664\tabularnewline
\textbf{AUC} & 0.993 & 0.954 & 0.996 & 0.965 & 0.966 &
0.860\tabularnewline
\bottomrule
\end{longtable}

Here, we describe a Python package called \texttt{demographicx} which
infers gender from first name and ethnicity from the full name. It is
based on fine-tuning a deep learning BERT embedding model with sub-word
tokenization (Devlin et al., 2018). Importantly, our model has the
ability to make predictions for names that it has not seen before. We
build our package on top of the popular transformers package, which
increases the likelihood that users will have parts of our models cached
in their computers. The dataset we used to train includes Genni + Ethnea
for the Author-ity 2009 dataset by Torvik (Torvik 2018), which has names
and predicted results by other previous methods. We mixed the dataset
with the Social Security Administration (SSA) popular newborn baby names
dataset (Social Security Administration 2013) and a Wikipedia name
ethnicity dataset (Ambekar et al. 2009). We validate our model with both
the aggregated data set and the Wikipedia datasets. Our models achieve
excellent performance on both tasks (see Table 1 and 2).

\begin{longtable}[]{@{}lllllllll@{}}
\caption{Race prediction performance on validation (val) split of the
mixed data set and Wikipedia (Wiki) names}\tabularnewline
\toprule
& Black & & Hispanic & & White & & Asian &\tabularnewline
\midrule
\endfirsthead
\toprule
& Black & & Hispanic & & White & & Asian &\tabularnewline
\midrule
\endhead
& Val & Wiki & Val & Wiki & Val & Wiki & Val & Wiki\tabularnewline
\textbf{F1} & 0.976 & 0.987 & 0.936 & 0.822 & 0.907 & 0.850 & 0.941 &
0.859\tabularnewline
\textbf{Acc} & 0.999 & 0.999 & 0.928 & 0.788 & 0.902 & 0.856 & 0.931 &
0.843\tabularnewline
\textbf{AUC} & 0.999 & 0.996 & 0.990 & 0.964 & 0.983 & 0.963 & 0.989 &
0.962\tabularnewline
\bottomrule
\end{longtable}

Because our package is built based on the \texttt{transformers} package,
it can be easily incorporated into PyTorch and transformers. The API is
very simple on purpose. Our package has already been used in (Acuna and
Liang 2021) and multiple other internal projects.

\begin{Shaded}
\begin{Highlighting}[]
\NormalTok{In: }\ImportTok{from}\NormalTok{ demographicx }\ImportTok{import}\NormalTok{ GenderEstimator}
\NormalTok{In: gender_estimator }\OperatorTok{=}\NormalTok{ GenderEstimator()}
\NormalTok{In: gender_estimator.predict(“Daniel”)}
\NormalTok{Out: \{‘female’: }\FloatTok{0.001}\NormalTok{, ‘male’: }\FloatTok{0.988}\NormalTok{, ‘unknown’, }\FloatTok{0.011}\NormalTok{\}}

\NormalTok{In: gender_estimator.predict(“Amy”)}
\NormalTok{Out: \{‘female’: }\FloatTok{0.998}\NormalTok{, ‘male’: }\FloatTok{0.001}\NormalTok{, ‘unknown’, }\FloatTok{0.001}\NormalTok{\}}

\NormalTok{In: }\ImportTok{from}\NormalTok{ demographicx }\ImportTok{import}\NormalTok{ EthnicityEstimator}
\NormalTok{In: ethnicity_estimator }\OperatorTok{=}\NormalTok{ EthnicityEstimator()}
\NormalTok{In: ethnicity_estimator.predict(“Daniel Acuna”)}
\NormalTok{Out: \{‘white’: }\FloatTok{0.002}\NormalTok{, ‘hispanic’: }\FloatTok{0.998}\NormalTok{, ‘black’, }\FloatTok{0.000}\NormalTok{, ‘asian’: }\FloatTok{0.000}\NormalTok{\}}

\NormalTok{In: ethnicity_estimator.predict(“Lizhen Liang”)}
\NormalTok{Out: \{‘white’: }\FloatTok{0.000}\NormalTok{, ‘hispanic’: }\FloatTok{0.000}\NormalTok{, ‘black’, }\FloatTok{0.000}\NormalTok{, ‘asian’: }\FloatTok{0.999}\NormalTok{\}}
\end{Highlighting}
\end{Shaded}

\hypertarget{acknowledgments}{%
\section{Acknowledgments}\label{acknowledgments}}

L. Liang and D. Acuna were partially funded by the National Science
Foundation grant ``Social Dynamics of Knowledge Transfer Through
Scientific Mentorship and Publication'' \#1933803. We thank Jim Yi for
his help with the repository.

\hypertarget{references}{%
\section*{References}\label{references}}
\addcontentsline{toc}{section}{References}

\hypertarget{refs}{}
\leavevmode\hypertarget{ref-acuna2020some}{}%
Acuna, Daniel E. 2020. ``Some Considerations for Studying Gender,
Mentorship, and Scientific Impact: Commentary on Alshebli, Makovi, and
Rahwan (2020).'' \emph{OSF Preprints}.
\url{https://doi.org/10.31219/osf.io/ybfk6}.

\leavevmode\hypertarget{ref-acuna2021are}{}%
Acuna, Daniel E, and Lizhen Liang. 2021. ``Are Ai Ethics Conferences
Different and More Diverse Compared to Traditional Computer Science
Conferences?'' In \emph{Fourth Aaai/Acm Conference on Artificial
Intelligence, Ethics, and Society (Aies'21)}. Association for Computing
Machinery. \url{https://doi.org/10.1145/3461702.3462616}.

\leavevmode\hypertarget{ref-ambekar2009name}{}%
Ambekar, Anurag, Charles Ward, Jahangir Mohammed, Swapna Male, and
Steven Skiena. 2009. ``Name-Ethnicity Classification from Open
Sources.'' In \emph{Proceedings of the 15th Acm Sigkdd International
Conference on Knowledge Discovery and Data Mining}, 49--58.
\url{https://doi.org/10.1145/1557019.1557032}.

\leavevmode\hypertarget{ref-ginther2011race}{}%
Ginther, Donna K, Walter T Schaffer, Joshua Schnell, Beth Masimore, Faye
Liu, Laurel L Haak, and Raynard Kington. 2011. ``Race, Ethnicity, and
Nih Research Awards.'' \emph{Science} 333 (6045). American Association
for the Advancement of Science: 1015--9.
\url{https://doi.org/10.1126/science.1196783}.

\leavevmode\hypertarget{ref-kozlowski2021avoiding}{}%
Kozlowski, Diego, Dakota S Murray, Alexis Bell, Will Hulsey, Vincent
Larivière, Thema Monroe-White, and Cassidy R Sugimoto. 2021. ``Avoiding
Bias When Inferring Race Using Name-Based Approaches.'' \emph{arXiv
Preprint arXiv:2104.12553}.

\leavevmode\hypertarget{ref-lariviere2013bibliometrics}{}%
Larivière, Vincent, Chaoqun Ni, Yves Gingras, Blaise Cronin, and Cassidy
R Sugimoto. 2013. ``Bibliometrics: Global Gender Disparities in
Science.'' \emph{Nature News} 504 (7479): 211.
\url{https://doi.org/10.1038/504211a}.

\leavevmode\hypertarget{ref-schwartz2021impact}{}%
Schwartz, Leah P, Jean Liénard, and Stephen V David. 2021. ``Impact of
Gender on the Formation and Outcome of Mentoring Relationships in
Academic Research.'' \emph{arXiv Preprint arXiv:2104.07780}.

\leavevmode\hypertarget{ref-social2013beyond}{}%
Social Security Administration. 2013. ``Beyond the Top 1000 Names.''
\emph{Retrieved March} 20: 2014.

\leavevmode\hypertarget{ref-illinoisdatabankIDB-9087546}{}%
Torvik, Vetle. 2018. ``Genni + Ethnea for the Author-Ity 2009 Dataset.''
University of Illinois at Urbana-Champaign.
\url{https://doi.org/10.13012/B2IDB-9087546_V1}.

\end{document}
